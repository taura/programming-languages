\documentclass[12pt,dvipdfmx]{beamer}
\usepackage{graphicx}
\DeclareGraphicsExtensions{.pdf}
\DeclareGraphicsExtensions{.eps}
\graphicspath{{out/}{out/tex/}{out/tex/gpl/}{out/tex/svg/}{out/tex/lsvg/}{out/tex/dot/}}
% \graphicspath{{out/}{out/tex/}{out/pdf/}{out/eps/}{out/tex/gpl/}{out/tex/svg/}{out/pdf/dot/}{out/pdf/gpl/}{out/pdf/img/}{out/pdf/odg/}{out/pdf/svg/}{out/eps/dot/}{out/eps/gpl/}{out/eps/img/}{out/eps/odg/}{out/eps/svg/}}
\usepackage{listings}
\usepackage{fancybox}
\usepackage{hyperref}
\usepackage{color}

%%%%%%%%%%%%%%%%%%%%%%%%%%%
%%% themes
%%%%%%%%%%%%%%%%%%%%%%%%%%%
\usetheme{default} % Szeged
%% no navigation bar
% default boxes Bergen Boadilla Madrid Pittsburgh Rochester
%% tree-like navigation bar
% Antibes JuanLesPins Montpellier
%% toc sidebar
% Berkeley PaloAlto Goettingen Marburg Hannover Berlin Ilmenau Dresden Darmstadt Frankfurt Singapore Szeged
%% Section and Subsection Tables
% Copenhagen Luebeck Malmoe Warsaw

%%%%%%%%%%%%%%%%%%%%%%%%%%%
%%% innerthemes
%%%%%%%%%%%%%%%%%%%%%%%%%%%
% \useinnertheme{circles}	% default circles rectangles rounded inmargin

%%%%%%%%%%%%%%%%%%%%%%%%%%%
%%% outerthemes
%%%%%%%%%%%%%%%%%%%%%%%%%%%
% outertheme
% \useoutertheme{default}	% default infolines miniframes smoothbars sidebar sprit shadow tree smoothtree


%%%%%%%%%%%%%%%%%%%%%%%%%%%
%%% colorthemes
%%%%%%%%%%%%%%%%%%%%%%%%%%%
\usecolortheme{seahorse}
%% special purpose
% default structure sidebartab 
%% complete 
% albatross beetle crane dove fly seagull 
%% inner
% lily orchid rose
%% outer
% whale seahorse dolphin

%%%%%%%%%%%%%%%%%%%%%%%%%%%
%%% fontthemes
%%%%%%%%%%%%%%%%%%%%%%%%%%%
\usefonttheme{serif}  
% default professionalfonts serif structurebold structureitalicserif structuresmallcapsserif

%%%%%%%%%%%%%%%%%%%%%%%%%%%
%%% generally useful beamer settings
%%%%%%%%%%%%%%%%%%%%%%%%%%%
% 
\AtBeginDvi{\special{pdf:tounicode EUC-UCS2}}
% do not show navigation
\setbeamertemplate{navigation symbols}{}
% show page numbers
\setbeamertemplate{footline}[frame number]

%%%%%%%%%%%%%%%%%%%%%%%%%%%
%%% define some colors for convenience
%%%%%%%%%%%%%%%%%%%%%%%%%%%

\newcommand{\mido}[1]{{\color{green}#1}}
\newcommand{\mura}[1]{{\color{purple}#1}}
\newcommand{\ore}[1]{{\color{orange}#1}}
\newcommand{\ao}[1]{{\color{blue}#1}}
\newcommand{\aka}[1]{{\color{red}#1}}

\setbeamercolor{ex}{bg=cyan!20!white}

%%%%%%%%%%%%%%%%%%%%%%%%%%%
%% customize beamer template
%% https://www.opt.mist.i.u-tokyo.ac.jp/~tasuku/beamer.html
%%%%%%%%%%%%%%%%%%%%%%%%%%%

\iffalse
%\renewcommand{\familydefault}{\sfdefault}  % 英文をサンセリフ体に
%\renewcommand{\kanjifamilydefault}{\gtdefault}  % 日本語をゴシック体に
\usefonttheme{structurebold} % タイトル部を太字
\setbeamerfont{alerted text}{series=\bfseries} % Alertを太字
\setbeamerfont{section in toc}{series=\mdseries} % 目次は太字にしない
\setbeamerfont{frametitle}{size=\Large} % フレームタイトル文字サイズ
\setbeamerfont{title}{size=\LARGE} % タイトル文字サイズ
\setbeamerfont{date}{size=\small}  % 日付文字サイズ

\definecolor{UniBlue}{RGB}{0,150,200} 
\definecolor{AlertOrange}{RGB}{255,76,0}
\definecolor{AlmostBlack}{RGB}{38,38,38}
\setbeamercolor{normal text}{fg=AlmostBlack}  % 本文カラー
\setbeamercolor{structure}{fg=UniBlue} % 見出しカラー
\setbeamercolor{block title}{fg=UniBlue!50!black} % ブロック部分タイトルカラー
\setbeamercolor{alerted text}{fg=AlertOrange} % \alert 文字カラー
\mode<beamer>{
    \definecolor{BackGroundGray}{RGB}{254,254,254}
    \setbeamercolor{background canvas}{bg=BackGroundGray} % スライドモードのみ背景をわずかにグレーにする
}


%フラットデザイン化
\setbeamertemplate{blocks}[rounded] % Blockの影を消す
\useinnertheme{circles} % 箇条書きをシンプルに
\setbeamertemplate{navigation symbols}{} % ナビゲーションシンボルを消す
\setbeamertemplate{footline}[frame number] % フッターはスライド番号のみ

%タイトルページ
\setbeamertemplate{title page}{%
    \vspace{2.5em}
    {\usebeamerfont{title} \usebeamercolor[fg]{title} \inserttitle \par}
    {\usebeamerfont{subtitle}\usebeamercolor[fg]{subtitle}\insertsubtitle \par}
    \vspace{1.5em}
    \begin{flushright}
        \usebeamerfont{author}\insertauthor\par
        \usebeamerfont{institute}\insertinstitute \par
        \vspace{3em}
        \usebeamerfont{date}\insertdate\par
        \usebeamercolor[fg]{titlegraphic}\inserttitlegraphic
    \end{flushright}
}
\fi

%%%%%%%%%%%%%%%%%%%%%%%%%%%
%%% how to typset code
%%%%%%%%%%%%%%%%%%%%%%%%%%%

\lstset{language = C,
numbers = left,
numberstyle = {\tiny \emph},
numbersep = 10pt,
breaklines = true,
breakindent = 40pt,
frame = tlRB,
frameround = ffft,
framesep = 3pt,
rulesep = 1pt,
rulecolor = {\color{blue}},
rulesepcolor = {\color{blue}},
flexiblecolumns = true,
keepspaces = true,
basicstyle = \ttfamily\scriptsize,
identifierstyle = ,
commentstyle = ,
stringstyle = ,
showstringspaces = false,
tabsize = 4,
escapechar=\@,
}

\title{Programming Languages (3) \\
  Going outside Jupyter \\
  and Using Libraries}
\institute{}
\author{Kenjiro Taura}
\date{}

\AtBeginSection[]
{
\begin{frame}
\frametitle{Contents}
\tableofcontents[currentsection]
\end{frame}
}

\iffalse
\AtBeginSubsection[]
{
\begin{frame}
\frametitle{Contents}
\tableofcontents[currentsection,currentsubsection]
\end{frame}
}
\fi

\newcommand{\therustrule}{single-owner-multiple-borrowers}

\begin{document}
\maketitle

%%%%%%%%%%%%%%%%%%%%%%%%%%%%%%%%%% 
% \begin{frame}
% \frametitle{Contents}
% \tableofcontents
% \end{frame}

%%%%%%%%%%%%%%%%% 
\begin{frame}
  \frametitle{Objectives}
  \begin{itemize}
  \item make programs outside Jupyter playground
    \begin{itemize}
    \item SSH (command line)
    \item editors, not web browsers
    \item build system
    \end{itemize}
  \item use libraries
  \item split a program into multiple files ($\approx$ use something defined in another file)
  \end{itemize}
\end{frame}

%%%%%%%%%%%%%%%%% 
\begin{frame}
  \frametitle{Build system}
  many languages have ``build system'' to help you
    use external libraries
  \begin{itemize}
  \item Go : {\tt go} is it
  \item Julia : no particular build system
  \item OCaml : {\tt dune} \url{https://dune.build/}
  \item Rust : {\tt cargo} 
  \end{itemize}
\end{frame}

%%%%%%%%%%%%%%%%% 
\begin{frame}
  \frametitle{Using libraries}
  using a library entails different procedures depending on how
  ``embedded'' it is into the language
  \begin{itemize}
  \item some libraries are \ao{\it ``builtin''}
    \begin{itemize}
    \item automatically available in every program
    \end{itemize}
  \item some libraries are \ao{\it ``standard''}
    \begin{itemize}
    \item you need to master how to refer to names in it
    \item you say ``import'' or ``use'' it
      and/or use prefixes to refer to names in it
    \item installed with the language
    \end{itemize}
    
  \item some libraries are \ao{\it ``external''}
    \begin{itemize}
    \item you may have to install it
    \item you may have to tell the compiler where it is
    \end{itemize}
  \end{itemize}
\end{frame}

  %%%%%%%%%%%%%%%%% 
\begin{frame}
  \frametitle{Importing a library to your program}
  \begin{itemize}
  \item assume $M$ is a module name and $n$ a name defined in $M$
  \item OCaml :
    \begin{itemize}
    \item call {\it M.n} 
    \item {\tt \ao{open} {\it M}} and call {\it n}
    \end{itemize}
    
  \item Julia : 
    \begin{itemize}
    \item {\tt \ao{import} {\it M}} and call {\it M.n} 
    \item {\tt \ao{using} {\it M}} and call {\it n}
    \end{itemize}

  \item Go :
    \begin{itemize}
    \item {\tt \ao{import} "{\it M}"} and call {\it M.n} 
    \end{itemize}

  \item Rust :
    \begin{itemize}
    \item a module may contain a module
    \item assume $C$ is the name of a ``crate''
    \item call {\tt {\it C}::$M_0$::$M_1$:: $\cdots$ ::{\it n}}
    \item {\tt \ao{use} {\it C}::$M_0$::$M_1$:: $\cdots$ ::{\it n}}
      and call \ao{\it n} 
    \item anywhere between the two
    \end{itemize}
  \end{itemize}
\end{frame}

%%%%%%%%%%%%%%%%% 
\begin{frame}
  \frametitle{Repository of libraries}
  \begin{itemize}
  \item master how to get information you need
    (names of functions, their types, etc.) from those repositories
  \item is it builtin? standard? external?
  \end{itemize}
  
  \begin{itemize}
  \item OCaml : opam \url{https://opam.ocaml.org/}
  \item Julia : Julia packages \url{https://julialang.org/packages/}
  \item Go : \url{https://pkg.go.dev/}
  \item Rust : \url{https://crates.io/}
  \end{itemize}
\end{frame}


\end{document}

